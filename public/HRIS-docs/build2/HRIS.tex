%% Generated by Sphinx.
\def\sphinxdocclass{report}
\documentclass[letterpaper,10pt,english]{sphinxmanual}
\ifdefined\pdfpxdimen
   \let\sphinxpxdimen\pdfpxdimen\else\newdimen\sphinxpxdimen
\fi \sphinxpxdimen=.75bp\relax

\PassOptionsToPackage{warn}{textcomp}
\usepackage[utf8]{inputenc}
\ifdefined\DeclareUnicodeCharacter
 \ifdefined\DeclareUnicodeCharacterAsOptional
  \DeclareUnicodeCharacter{"00A0}{\nobreakspace}
  \DeclareUnicodeCharacter{"2500}{\sphinxunichar{2500}}
  \DeclareUnicodeCharacter{"2502}{\sphinxunichar{2502}}
  \DeclareUnicodeCharacter{"2514}{\sphinxunichar{2514}}
  \DeclareUnicodeCharacter{"251C}{\sphinxunichar{251C}}
  \DeclareUnicodeCharacter{"2572}{\textbackslash}
 \else
  \DeclareUnicodeCharacter{00A0}{\nobreakspace}
  \DeclareUnicodeCharacter{2500}{\sphinxunichar{2500}}
  \DeclareUnicodeCharacter{2502}{\sphinxunichar{2502}}
  \DeclareUnicodeCharacter{2514}{\sphinxunichar{2514}}
  \DeclareUnicodeCharacter{251C}{\sphinxunichar{251C}}
  \DeclareUnicodeCharacter{2572}{\textbackslash}
 \fi
\fi
\usepackage{cmap}
\usepackage[T1]{fontenc}
\usepackage{amsmath,amssymb,amstext}
\usepackage{babel}
\usepackage{times}
\usepackage[Bjarne]{fncychap}
\usepackage{sphinx}

\usepackage{geometry}

% Include hyperref last.
\usepackage{hyperref}
% Fix anchor placement for figures with captions.
\usepackage{hypcap}% it must be loaded after hyperref.
% Set up styles of URL: it should be placed after hyperref.
\urlstyle{same}
\addto\captionsenglish{\renewcommand{\contentsname}{Contents:}}

\addto\captionsenglish{\renewcommand{\figurename}{Fig.}}
\addto\captionsenglish{\renewcommand{\tablename}{Table}}
\addto\captionsenglish{\renewcommand{\literalblockname}{Listing}}

\addto\captionsenglish{\renewcommand{\literalblockcontinuedname}{continued from previous page}}
\addto\captionsenglish{\renewcommand{\literalblockcontinuesname}{continues on next page}}

\addto\extrasenglish{\def\pageautorefname{page}}

\setcounter{tocdepth}{2}



\title{HRIS Documentation}
\date{Jun 08, 2018}
\release{May 21, 2018}
\author{Pushpa Nyaupane}
\newcommand{\sphinxlogo}{\vbox{}}
\renewcommand{\releasename}{Release}
\makeindex

\begin{document}

\maketitle
\sphinxtableofcontents
\phantomsection\label{\detokenize{HRIS::doc}}



\chapter{Employee}
\label{\detokenize{employee/employee:employee}}\label{\detokenize{employee/employee:hris-documentation}}\label{\detokenize{employee/employee::doc}}
You can add, edit or delete employee records from the system using the employee tab.


\section{Creating an Employee}
\label{\detokenize{employee/add:creating-an-employee}}\label{\detokenize{employee/add::doc}}\label{\detokenize{employee/add:add-employee}}
For adding an employee, you will have to navigate to employees tab, then click on New button displayed on the top-right section of the screen.

\begin{figure}[htbp]
\centering
\capstart

\noindent\sphinxincludegraphics[scale=0.5]{{add-new-button}.png}
\caption{Click on the New Button to add a new employee.}\label{\detokenize{employee/add:id3}}\end{figure}

Once the add new employee panel opens up, you’ll have to fill in the form. Some fields are mandatory(marked with a {\color{red}\bfseries{}*}) and some are optional. There are multiple tabs presented in then panel, which are to be filled in a serial manner.

\begin{figure}[htbp]
\centering
\capstart

\noindent\sphinxincludegraphics[scale=0.5]{{add-tabs}.png}
\caption{various tabs while adding an employee.}\label{\detokenize{employee/add:id4}}\end{figure}


\subsection{General Info}
\label{\detokenize{employee/add:general-info}}
This section collects the basic details for the employee.

\begin{figure}[htbp]
\centering
\capstart

\noindent\sphinxincludegraphics[scale=0.5]{{add-general}.png}
\caption{adding general info about an employee.}\label{\detokenize{employee/add:id5}}\end{figure}


\subsection{Family Details}
\label{\detokenize{employee/add:family-details}}
This section collects the details related to the family of an employee.

\begin{figure}[htbp]
\centering
\capstart

\noindent\sphinxincludegraphics[scale=0.5]{{add-family}.png}
\caption{adding general info about an employee.}\label{\detokenize{employee/add:id6}}\end{figure}


\subsection{Employee Identification}
\label{\detokenize{employee/add:employee-identification}}
This section collects the identification details related to the employee.

\begin{figure}[htbp]
\centering
\capstart

\noindent\sphinxincludegraphics[scale=0.5]{{add-identification}.png}
\caption{adding general info about an employee.}\label{\detokenize{employee/add:id7}}\end{figure}


\subsection{Status \& Integration}
\label{\detokenize{employee/add:status-integration}}
This section collects info about employee position in the company, designation, location, salary details, and data about reporting hierarchy.

\begin{figure}[htbp]
\centering
\capstart

\noindent\sphinxincludegraphics[scale=0.5]{{add-status}.png}
\caption{Status and Integration Tab while adding an employee.}\label{\detokenize{employee/add:id8}}\end{figure}


\subsection{Qualification}
\label{\detokenize{employee/add:qualification}}
This section captures Academic Degrees/Qualifications about an employee. One has to configure academic qualifications parameters in the master setup so that the options for selecting universities/degrees etc. appear in this menu.

\begin{figure}[htbp]
\centering
\capstart

\noindent\sphinxincludegraphics[scale=0.5]{{add-qualification}.png}
\caption{Adding qualifications details while creating an employee.}\label{\detokenize{employee/add:id9}}\end{figure}


\subsection{Document Upload}
\label{\detokenize{employee/add:document-upload}}
This section enables an organization to upload the employee picture along with relevant documents.

\begin{figure}[htbp]
\centering
\capstart

\noindent\sphinxincludegraphics[scale=0.5]{{add-documents}.png}
\caption{Uploading employee profile image and relevant documents.}\label{\detokenize{employee/add:id10}}\end{figure}


\subsection{Experience}
\label{\detokenize{employee/add:experience}}
This section collects past experiences of an employee.

\begin{figure}[htbp]
\centering
\capstart

\noindent\sphinxincludegraphics[scale=0.5]{{add-experience}.png}
\caption{Storing past experiences of an employee.}\label{\detokenize{employee/add:id11}}\end{figure}


\subsection{Training}
\label{\detokenize{employee/add:training}}
This section collects trainings that an employee has done.

\begin{figure}[htbp]
\centering
\capstart

\noindent\sphinxincludegraphics[scale=0.5]{{add-training}.png}
\caption{Storing training of an employee.}\label{\detokenize{employee/add:id12}}\end{figure}


\subsection{Additional Information}
\label{\detokenize{employee/add:additional-information}}
This lets an organization assign leaves, trainings and appraisal while adding an employee.

\begin{figure}[htbp]
\centering
\capstart

\noindent\sphinxincludegraphics[scale=0.5]{{add-extra}.png}
\caption{Storing training of an employee.}\label{\detokenize{employee/add:id13}}\end{figure}


\section{Managing Employees}
\label{\detokenize{employee/view::doc}}\label{\detokenize{employee/view:managing-employees}}
When you open the employee tab, you are presented with a form which lets you search for employees using various filters.

\begin{figure}[htbp]
\centering
\capstart

\noindent\sphinxincludegraphics[scale=0.5]{{view-filters}.png}
\caption{various filters for viewing employee.}\label{\detokenize{employee/view:id1}}\end{figure}

These filters let you view employees from specific company, branch, departments, designation, position, service type, service event type, employee types, gender, location or individual employee.

If you do not apply any filters and press Search button, it lists out all the employees that the logged in person has access to. HRs have company-wide access to employee data.

After clicking search, you are presented with list of employees matching the filters.

\begin{figure}[htbp]
\centering
\capstart

\noindent\sphinxincludegraphics[scale=0.5]{{view-employeeList}.png}
\caption{Matched employeed with applied filters.}\label{\detokenize{employee/view:id2}}\end{figure}

With every employees matching the filters, there comes data associated with each employee in a tabular format. Also, every employee record has 3 buttons displayed:

\begin{figure}[htbp]
\centering
\capstart

\noindent\sphinxincludegraphics[scale=0.5]{{view-employee-buttons}.png}
\caption{View,edit and delete buttons(from left to right) displayed with each records.}\label{\detokenize{employee/view:id3}}\end{figure}


\subsection{View Details}
\label{\detokenize{employee/view:view-details}}
This button allows to quickly view all details about an employee. e.g. name, address, position, history, trainings, service status, assets given to the employee, etc.

\begin{figure}[htbp]
\centering
\capstart

\noindent\sphinxincludegraphics[scale=0.5]{{employee/img/view-employee}.png}
\caption{View,edit and delete buttons(from left to right) displayed with each records.}\label{\detokenize{employee/view:id4}}\end{figure}


\subsection{Edit Employee}
\label{\detokenize{employee/view:edit-employee}}
The edit option lets users to update data about the employee, using the exact form and options that are used when {\hyperref[\detokenize{employee/add:add-employee}]{\sphinxcrossref{\DUrole{std,std-ref}{Creating an Employee}}}}


\subsection{Delete Employee Record}
\label{\detokenize{employee/view:delete-employee-record}}
With this button, you can delete an employee record so that it doesnt appear in the employee list. But this is not the suggested way to retire an employee. This is only useful when you want to delete an erroneous record created without intention of actually using them e.g test users. Deleting an record from this panel might result in inconsistency across other modules. You can retire an employee using \DUrole{xref,std,std-ref}{retire an employee}


\chapter{Appraisal}
\label{\detokenize{appraisal/appraisal:appraisal}}\label{\detokenize{appraisal/appraisal::doc}}

\section{Creating Questionnaire}
\label{\detokenize{appraisal/add::doc}}\label{\detokenize{appraisal/add:creating-questionnaire}}
Creating a questionnaire for appraisal purpose is made easy with neo.HRIS. One can easily create and assign various questions that can be configured to be asked to specific personnel e.g. appraisee, recommendor, approver or HR.
\begin{itemize}
\item {} 
A thing.

\item {} 
Another thing.

\end{itemize}


\section{Assigning Appraisal}
\label{\detokenize{appraisal/view:assigning-appraisal}}\label{\detokenize{appraisal/view::doc}}
Once you have created a questionnaire, its very easy to assign the same to the related people. You can either assign to multiple people in bulk using filters or individually.


\chapter{Attendance}
\label{\detokenize{attendance/attendance:attendance}}\label{\detokenize{attendance/attendance::doc}}\label{\detokenize{attendance/attendance:id1}}
Our system provides features to configure and monitor the attendance of employees in the organization. There are various reports generated on the basis of the configured parameters. The system is also able to calculate OT automatically when configured.


\section{Attendance Report}
\label{\detokenize{attendance/report:attendance-report}}\label{\detokenize{attendance/report::doc}}\label{\detokenize{attendance/report:id1}}
The attendance report allows a manager or HR personnel to quickly view attendance for employees. The attendance report is controlled with the access control logic so that one can only see the reports for the employees whose data is accessible for them.
Attendance report is visible for managers from the Manager Service Menu and from the HR menu for HR personnel.


\subsection{Manager Report}
\label{\detokenize{attendance/report:manager-report}}\label{\detokenize{attendance/report:manager-attendance-report}}
\begin{figure}[htbp]
\centering
\capstart

\noindent\sphinxincludegraphics[scale=0.5]{{report-manager-entry}.png}
\caption{Viewing report from the manager service menu.}\label{\detokenize{attendance/report:id2}}\end{figure}

Managers can put a filter for date range, specific employee and attendance status e.g. Present, Absent, Missed Punch, On Holiday, Late in/ Early Out, etc.


\subsection{HR Report}
\label{\detokenize{attendance/report:hr-report}}
\begin{figure}[htbp]
\centering
\capstart

\noindent\sphinxincludegraphics[scale=0.5]{{report-hr-entry}.png}
\caption{Viewing report from the HR menu.}\label{\detokenize{attendance/report:id3}}\end{figure}


\section{Troubleshooting Attendance}
\label{\detokenize{attendance/troubleshooting:troubleshooting-attendance}}\label{\detokenize{attendance/troubleshooting::doc}}\label{\detokenize{attendance/troubleshooting:attendance-troubleshooting}}
Since attendance is a essential component of daily operations, you can face issues in the attendance from time to time. We have built various tools that can assist the users to troubleshoot such problems easily.


\subsection{ZKTeco Attendance Manager}
\label{\detokenize{attendance/troubleshooting:zkteco-attendance-manager}}
Our compatiable attendance devices come up with a software built separately to track individual attendance data stored on the devices. In case you see some attendance data not appearing in the attendance report, this is the first place you should look at. If the employee has made a thumb/card punch, it should be present in this software. If the data doesnt exist in this software, it will not be sent to the software either. This means the attendance in question was not registered by the device itself.

You can view attendance in this software by obtaining the thumb-id assigned for the employee from their profile, and IP Address of the attendance device.

If you see the attendance data in the ZKT software, but not on HRIS, you should follow other troubleshooting methods described  below.


\subsection{Invalid Join Date}
\label{\detokenize{attendance/troubleshooting:invalid-join-date}}
Another common case is when an employee is assigned an invalid joining date e.g empty value or a date in the future. As our system logs attendance for an user only after the joining date, having invalid joining date results in no attendance data.

If you have this problem and changed the joining date, you’ll need to run reattendance for the user to obtain past attendance records.


\subsection{Duplicate / Empty Thumb ID}
\label{\detokenize{attendance/troubleshooting:duplicate-empty-thumb-id}}
Another common case is when an employee is assigned empty/duplicate Thumb ID. Having duplicate thumb id causes conflict with other data, and having no thumb id means the system wouldnt be able to identify who has punched in.

Once fixing this problem, you’ll need to run reattendance for the user to view past attendance data.


\section{Re Attendance}
\label{\detokenize{attendance/troubleshooting:re-attendance}}
The reattendance tool makes it easy to re run the attendance procedure for selected employees. This is useful when you want the system to reload attendance after changing certain user parameters while troubleshooting.

\begin{figure}[htbp]
\centering
\capstart

\noindent\sphinxincludegraphics[scale=0.5]{{trouble-reattendance}.png}
\caption{Reattendance menu.}\label{\detokenize{attendance/troubleshooting:id1}}\end{figure}

\begin{figure}[htbp]
\centering
\capstart

\noindent\sphinxincludegraphics[scale=0.5]{{trouble-reattendance-tool}.png}
\caption{Reattendance tool.}\label{\detokenize{attendance/troubleshooting:id2}}\end{figure}

You’ll have to specify the date range, and select employees for whom you want to redo the attendance procedure. Once done so, click on submit and the tool does the attendance procedure.

You can verify the re attendance using the attendance report.


\section{Overtime}
\label{\detokenize{attendance/overtime:overtime}}\label{\detokenize{attendance/overtime::doc}}\label{\detokenize{attendance/overtime:id1}}
With neo-hris, its very simple to manage overtime work. Once you have assigned overtime to employees, it tracks their overtime work automatically.


\chapter{Leave/Holiday}
\label{\detokenize{leave-holiday/leave-holiday:leave-holiday}}\label{\detokenize{leave-holiday/leave-holiday::doc}}\label{\detokenize{leave-holiday/leave-holiday:id1}}
Employees are provided with various leave benefits to allow time for rest and recreation and to provide time off for personal and emergency purposes.
The software provides a convenient way to create and assign various types of leaves/holidays that are applicable in the organization.


\section{Leave categorization}
\label{\detokenize{leave-holiday/leave-types::doc}}\label{\detokenize{leave-holiday/leave-types:leave-categorization}}
Generally leaves are categorized under 2 types:


\subsection{Recurring Leaves}
\label{\detokenize{leave-holiday/leave-types:recurring-leaves}}
Leave which are entitled in each leave fiscal year. Financial Fiscal Year may be different than the leave fiscal year.
\begin{itemize}
\item {} 
Casual Leave

\item {} 
Home Leave

\item {} 
Sick Leave

\item {} 
Annual Leave

\end{itemize}


\subsection{Service Period Leaves}
\label{\detokenize{leave-holiday/leave-types:service-period-leaves}}
Some leaves are to be provided to employees only in certain situations.
\begin{itemize}
\item {} 
Maternity Leave

\item {} 
Paternity Leave

\item {} 
Mourning Leave

\end{itemize}


\section{Leave Record Management}
\label{\detokenize{leave-holiday/leave-configuration:leave-record-management}}\label{\detokenize{leave-holiday/leave-configuration::doc}}
Leave Management system would store the leave consumed by employees of the organization with specific details about leave types. It should consider inclusion of at least the following facts:
\begin{itemize}
\item {} 
Leave Type

\item {} 
Entitlement

\item {} 
Consumed Leave  on particular duration (week, month, year)

\item {} 
Earned Balance till yet

\item {} 
Claimable Balance till yet

\item {} 
Cashable Balance until yet

\end{itemize}


\subsection{Leave Configuration}
\label{\detokenize{leave-holiday/leave-configuration:leave-configuration}}
Configuration of each leave type should be parameterized with the following considerations:
\begin{itemize}
\item {} 
Applicable Leave Types: Any particular leave type might not be applicable to all employees. Hence, its associated applicable leave types should be always definable.

\item {} 
Minimum Avail Duration: Half Day/Full Day

\item {} 
Carry Forward: Defining whether any particular leave type can be carried forward to another fiscal year or not, if yes, is there any limit to the maximum accumulation or not.

\item {} 
Holiday/Week off Exclusion: Yes/No

\end{itemize}


\subsection{Leave Setup in HRIS}
\label{\detokenize{leave-holiday/leave-configuration:leave-setup-in-hris}}
You can access the leave option under the setup menu.

\begin{figure}[htbp]
\centering
\capstart

\noindent\sphinxincludegraphics[scale=0.5]{{add-leave-menu}.png}
\caption{Firstly open leave setup from setup menu in HRIS then click on New button on top-right.}\label{\detokenize{leave-holiday/leave-configuration:id1}}\end{figure}

\begin{figure}[htbp]
\centering
\capstart

\noindent\sphinxincludegraphics[scale=0.5]{{add-leave-panel}.png}
\caption{Enter data regarding the leave in HRIS using the below parameters and submit.}\label{\detokenize{leave-holiday/leave-configuration:id2}}\end{figure}
\begin{itemize}
\item {} 
\sphinxstylestrong{Leave Ename:} This field Specify the Name of Leave.

\item {} 
\sphinxstylestrong{Fiscal Year:} The created leave is applicable for which fiscal year is determined by this field.

\item {} 
\sphinxstylestrong{Default Days:} The default days for created leave are defined by this field. Actually it is defined by the company’s rule.

\item {} 
\sphinxstylestrong{Max Accumulate Days:} This filed specify the number of maximum accumulated days for the created leave which  should be carry forward or paid.

\item {} 
\sphinxstylestrong{Paid:} This field specifies whether the created leave is paid leave or not.

\item {} 
\sphinxstylestrong{Cashable:} This field specifies whether the max accumulated days is cashable or not at the end of fiscal year or not.

\item {} 
\sphinxstylestrong{Carry forward:} This field specifies where the max accumulated days are Carried forward or not at the end of Fiscal Year.

\item {} 
\sphinxstylestrong{Is Substitute Mandatory:} This field determines whether the substitute Co-worker is compulsory required for the created leave or not.

\item {} 
\sphinxstylestrong{Allow Half day:} This field determines whether half day leave is allowable for the created leave or not.

\item {} 
\sphinxstylestrong{Allow Grace Leave:}  This field determines whether quarter leave of created leave is allowed or not i.e. one fourth of total working hour is allowed or not.

\item {} 
\sphinxstylestrong{Is Monthly:} This field determines whether the leave is applicable for each month or not.

\item {} 
\sphinxstylestrong{Assign on Employee setup:} This field determines whether the created leave should be automatically assigned on employee creation time or not.

\item {} 
\sphinxstylestrong{Is Prorata Basis:} This field determines whether the leave is applicable on pro rata basis or not. For e.g. if casual leave is assigned 12 days for a fiscal year \& employee gets enrolled in the mid of fiscal year then he/she is applicable to get 6 days or not.

\item {} 
\sphinxstylestrong{Remarks:} Remarks can be added on this field.

\end{itemize}


\subsection{Leave Filters}
\label{\detokenize{leave-holiday/leave-configuration:leave-filters}}
Filters for assigning leaves is present below the leave creation form.

\begin{figure}[htbp]
\centering
\capstart

\noindent\sphinxincludegraphics[scale=0.5]{{add-leave-filters}.png}
\caption{Assigning leaves using a filter while creation.}\label{\detokenize{leave-holiday/leave-configuration:id3}}\end{figure}

The system allows to create and apply leaves to a filtered group of employees while creating it. You can make a leave specific to any combination of the following:
\begin{itemize}
\item {} 
Company

\item {} 
Branch

\item {} 
Department

\item {} 
Designation

\item {} 
Position

\item {} 
Service Type

\item {} 
Service Event Type

\item {} 
Employee Type

\item {} 
Employee

\item {} 
Gender

\end{itemize}


\subsection{Manual Leave Assignment}
\label{\detokenize{leave-holiday/leave-configuration:manual-leave-assignment}}
In case when you want to assign leaves manually to users, you can do it from the assign panel.

First you’ll need to navigate to assign tab on the sidebar and click Leaves under this menu.

\begin{figure}[htbp]
\centering
\capstart

\noindent\sphinxincludegraphics[scale=0.5]{{assign-leave-menu}.png}
\caption{Assigning leaves to employees.}\label{\detokenize{leave-holiday/leave-configuration:id4}}\end{figure}

Then you can apply filters to list emplpyees as desired and then select the leave that you want to assign to them from the filter bar.

\begin{figure}[htbp]
\centering
\capstart

\noindent\sphinxincludegraphics[scale=0.5]{{assign-leave}.png}
\caption{Assigning leaves to employees while creation.}\label{\detokenize{leave-holiday/leave-configuration:id5}}\end{figure}

You can then tick employees that you want to assign the leave or use bulk select all using the select bar on table header. Once done so, you’ll need to enter previous leave balance on the left box appearing below the table and number of leaves to be assigned on the right textbox. Then press submit button to assign the leave details to selected employees.


\section{Leave Balances}
\label{\detokenize{leave-holiday/leave-balance::doc}}\label{\detokenize{leave-holiday/leave-balance:leave-balances}}
HRIS provides a convenient way to view and manage leave balances for the organization. Employees can view their personal leave balances using the self service portal. For HR and admin, we provide a tool so that its easy and convenient to see leave balance in bulk.

\begin{figure}[htbp]
\centering
\capstart

\noindent\sphinxincludegraphics[scale=0.5]{{menu-leave-balance}.png}
\caption{Leave Balance Menu.}\label{\detokenize{leave-holiday/leave-balance:id1}}\end{figure}

\begin{figure}[htbp]
\centering
\capstart

\noindent\sphinxincludegraphics[scale=0.5]{{view-leave-balance}.png}
\caption{Leave Balance Panel.}\label{\detokenize{leave-holiday/leave-balance:id2}}\end{figure}


\chapter{System Setup}
\label{\detokenize{setup/setup::doc}}\label{\detokenize{setup/setup:system-setup}}

\chapter{Manager Service}
\label{\detokenize{manager/portal:manager-service}}\label{\detokenize{manager/portal::doc}}\label{\detokenize{manager/portal:id1}}
The manager service portal makes it easy to do  managerial activities e.g. request approval, status reporting, attendance reports, appraisal evaluation etc.


\section{Request Approval}
\label{\detokenize{manager/manager-approval::doc}}\label{\detokenize{manager/manager-approval:request-approval}}\label{\detokenize{manager/manager-approval:manager-approval}}
The request approval panel for managers provides an easy way to manage approval process for the managers. The approval requests can range from leave, loan, travel - expense, training overtime, etc.

\begin{figure}[htbp]
\centering
\capstart

\noindent\sphinxincludegraphics[scale=0.5]{{manager-approval-menu}.PNG}
\caption{Various approval tasks for a Manager}\label{\detokenize{manager/manager-approval:id4}}\end{figure}


\subsection{Request Notification}
\label{\detokenize{manager/manager-approval:request-notification}}
Whenever a subordinate makes a request that requires manager approval, a notification is sent to their managers right away. When the manager clicks on the notification, it opens the confirmation dialog for quickly recommending/approving/rejecting the request.

\begin{figure}[htbp]
\centering
\capstart

\noindent\sphinxincludegraphics[scale=0.5]{{leave-manager-notification}.PNG}
\caption{A leave notification sent to the manager.}\label{\detokenize{manager/manager-approval:id5}}\end{figure}

Also, on approval/reject of the request, a  notification is also sent to the requestor so that they can quickly get notified about their request.

Alongside notifications in HRIS, it’s also possible to trigger emails on such request-approval processes. For doing so, the software requires one time set up of the email template. For more details, refer to :ref: \sphinxtitleref{Email Setup} docs.


\subsection{Leave Appproval}
\label{\detokenize{manager/manager-approval:leave-appproval}}
The leave approval portal lets a manager quickly view and approve the leave requests and their status.

\begin{figure}[htbp]
\centering
\capstart

\noindent\sphinxincludegraphics[scale=0.5]{{manager-leave-approval}.PNG}
\caption{Various approval tasks for a Manager}\label{\detokenize{manager/manager-approval:id6}}\end{figure}

In the approval panel, you are presented with all the requests made by your subordinates. For recommenders, they have an option to recommend/reject the leave requests while the approvers have the ability to approve the requests.

For approving/rejecting a request, click on the \sphinxstyleemphasis{} icon on right side of a request.

\begin{figure}[htbp]
\centering
\capstart

\noindent\sphinxincludegraphics[scale=0.5]{{manager-leave-confirmation}.PNG}
\caption{Approving/Rejecting a leave request}\label{\detokenize{manager/manager-approval:id7}}\end{figure}

From this dialog, you can either approve or reject the request. In case where you want to write a note regarding your decision,


\subsection{Attendance Approval}
\label{\detokenize{manager/manager-approval:attendance-approval}}\label{\detokenize{manager/manager-approval:id1}}
In cases where an employee misses an attendance punch/web-punch, they have to request it manually. Once they have made a request, a notification is sent and an entry in the attendance approval panel for the manager is created. A manager needs to approve the attendance request so that the attendance is marked in the database is Present.

For viewing the attendance requests, navigate to Attendance to Approve menu under the Manager Service menu.

\begin{figure}[htbp]
\centering
\capstart

\noindent\sphinxincludegraphics[scale=0.5]{{manager-attendance-menu}.PNG}
\caption{Attendance Approval Menu for Managers.}\label{\detokenize{manager/manager-approval:id8}}\end{figure}

\begin{figure}[htbp]
\centering
\capstart

\noindent\sphinxincludegraphics[scale=0.5]{{manager-attendance-approval}.PNG}
\caption{Attendance Requests for Managers.}\label{\detokenize{manager/manager-approval:id9}}\end{figure}

For viewing the details of a attendance request, click on the \sphinxstyleemphasis{} icon on right side of a request.

\begin{figure}[htbp]
\centering
\capstart

\noindent\sphinxincludegraphics[scale=0.5]{{manager-attendance-confirmation}.PNG}
\caption{Approving an attendance request.}\label{\detokenize{manager/manager-approval:id10}}\end{figure}


\subsection{Loan Approval}
\label{\detokenize{manager/manager-approval:id2}}\label{\detokenize{manager/manager-approval:loan-approval}}
For viewing the Loan requests, navigate to Loan to Approve menu under the Manager Service menu.

\begin{figure}[htbp]
\centering
\capstart

\noindent\sphinxincludegraphics[scale=0.5]{{manager-loan-approval}.PNG}
\caption{Attendance Requests for Managers.}\label{\detokenize{manager/manager-approval:id11}}\end{figure}

For viewing the details of a loan request, click on the \sphinxstyleemphasis{} icon on right side of the request.

\begin{figure}[htbp]
\centering
\capstart

\noindent\sphinxincludegraphics[scale=0.5]{{manager-loan-confirmation}.PNG}
\caption{Approving an loan request.}\label{\detokenize{manager/manager-approval:id12}}\end{figure}


\subsection{Travel/Expense Approval}
\label{\detokenize{manager/manager-approval:travel-expense-approval}}
There are two types of travel requests, a travel request and an expense request. Both appear under the same menu for approval, with a parameter called “Request for” determining whether the request is for travel or expense. An expense request can only be added once a travel request is made.

For viewing the travel/expense requests, navigate to Travel to Approve menu under the Manager Service menu.

\begin{figure}[htbp]
\centering
\capstart

\noindent\sphinxincludegraphics[scale=0.5]{{manager-travel-approval}.PNG}
\caption{Travel requests for Managers.}\label{\detokenize{manager/manager-approval:id13}}\end{figure}

For viewing the details of a travel/expense request, click on the \sphinxstyleemphasis{} icon on right side of the request.

\begin{figure}[htbp]
\centering
\capstart

\noindent\sphinxincludegraphics[scale=0.5]{{manager-travel-confirmation}.PNG}
\caption{Approving an travel request.}\label{\detokenize{manager/manager-approval:id14}}\end{figure}

\begin{figure}[htbp]
\centering
\capstart

\noindent\sphinxincludegraphics[scale=0.5]{{manager-expense-confirmation}.PNG}
\caption{Approving an expense request.}\label{\detokenize{manager/manager-approval:id15}}\end{figure}


\subsection{Work on Day Off}
\label{\detokenize{manager/manager-approval:work-on-day-off}}\label{\detokenize{manager/manager-approval:id3}}
For approving the work on day off requests, navigate to the Work on Day Off to Approve menu under Manager Service.

\begin{figure}[htbp]
\centering
\capstart

\noindent\sphinxincludegraphics[scale=0.5]{{manager-work-dayoff-request}.PNG}
\caption{Viewing Work on Day off Requests.}\label{\detokenize{manager/manager-approval:id16}}\end{figure}

For viewing the details of a work on day off request, click on the \sphinxstyleemphasis{} icon on right side of the request. Or you can click on the notification to directly see the details of the request.

\begin{figure}[htbp]
\centering
\capstart

\noindent\sphinxincludegraphics[scale=0.5]{{manager-work-dayoff-confirmation}.PNG}
\caption{Approving work on day off request.}\label{\detokenize{manager/manager-approval:id17}}\end{figure}


\subsection{Work On Holiday}
\label{\detokenize{manager/manager-approval:work-on-holiday}}
For approving the work on holiday requests, You can navigate to Work on Holiday to Approve menu under Manager Service Menu.

\begin{figure}[htbp]
\centering
\capstart

\noindent\sphinxincludegraphics[scale=0.5]{{manager-woh-requests}.PNG}
\caption{Viewing work on holiday Requests.}\label{\detokenize{manager/manager-approval:id18}}\end{figure}

For viewing the details of a work on holiday request, click on the \sphinxstyleemphasis{} icon on right side of the request. Or you can click on the notification to directly see the details of the request.

\begin{figure}[htbp]
\centering
\capstart

\noindent\sphinxincludegraphics[scale=0.5]{{manager-woh-confirmation}.PNG}
\caption{Approving work on holiday request.}\label{\detokenize{manager/manager-approval:id19}}\end{figure}


\subsection{Training Approval}
\label{\detokenize{manager/manager-approval:training-approval}}
For approving the training requests, You can navigate toTraining to Approve menu under Manager Service Menu.

\begin{figure}[htbp]
\centering
\capstart

\noindent\sphinxincludegraphics[scale=0.5]{{manager-training-requests}.PNG}
\caption{Viewing Training Requests.}\label{\detokenize{manager/manager-approval:id20}}\end{figure}

For viewing the details of a training request, click on the \sphinxstyleemphasis{} icon on right side of the request. Or you can click on the notification to directly see the details of the request.

\begin{figure}[htbp]
\centering
\capstart

\noindent\sphinxincludegraphics[scale=0.5]{{manager-training-confirmation}.PNG}
\caption{Approving a training request.}\label{\detokenize{manager/manager-approval:id21}}\end{figure}


\subsection{Overtime Approval}
\label{\detokenize{manager/manager-approval:overtime-approval}}
For approving the work on holiday requests, You can navigate to Overtime to Approve menu under Manager Service Menu.

\begin{figure}[htbp]
\centering
\capstart

\noindent\sphinxincludegraphics[scale=0.5]{{manager-overtime-requests}.PNG}
\caption{Viewing Overtime Requests.}\label{\detokenize{manager/manager-approval:id22}}\end{figure}

For viewing the details of a overtime request, click on the \sphinxstyleemphasis{} icon on right side of the request. Or you can click on the notification to directly see the details of the request.

\begin{figure}[htbp]
\centering
\capstart

\noindent\sphinxincludegraphics[scale=0.5]{{manager-overtime-confirmation}.PNG}
\caption{Approving a overtime request.}\label{\detokenize{manager/manager-approval:id23}}\end{figure}


\section{Subordinate Attendance Report}
\label{\detokenize{manager/attendance-report:subordinate-attendance-report}}\label{\detokenize{manager/attendance-report::doc}}
Attendance report for manager provides a quick way to oversee the attendance of employees under their management. A manager may use various filters to see attendance report of their preferred employee/a group.

\begin{figure}[htbp]
\centering
\capstart

\noindent\sphinxincludegraphics[scale=0.5]{{manager-attendance-report}.PNG}
\caption{Attendance report for a Manager}\label{\detokenize{manager/attendance-report:id1}}\end{figure}

You can use various filters e.g. based on date, employee name, attendance status(Present, absent, on leave/training etc). Also a manager can click missed punch only to see relevant data.


\section{Subordinate Request Status}
\label{\detokenize{manager/status-reporting:subordinate-request-status}}\label{\detokenize{manager/status-reporting::doc}}
The status reporting portal lets a manager quickly see request status regarding employees under their management. A manager can quickly go through the requests for attendance, leave, travel, training, overtime, work on holiday, loan etc.  relating to their subordinates.

These reports are similar to what we see in the approval panel, but the approval panel only lists the requests for which there is a decision pending. Once the approval/rejection is done, it doesnt appear in the Approval Panel. They are now moved to the status reports. One can see the request details and historical data in the status reports.

\begin{figure}[htbp]
\centering
\capstart

\noindent\sphinxincludegraphics[scale=1.5]{{manager-status-menus}.PNG}
\caption{Various status reports on the manager service menu.}\label{\detokenize{manager/status-reporting:id1}}\end{figure}

Most of the status reports come with a filter bar, that lets the managers quickly filter employees combining filters for company, branch, department, service status, request status(pending/recommended/approved/rejected), types specific to the requests, date ranges, etc.

\begin{figure}[htbp]
\centering
\capstart

\noindent\sphinxincludegraphics[scale=0.5]{{manager-status-filter-ex}.PNG}
\caption{Filtering options presented in a status report.}\label{\detokenize{manager/status-reporting:id2}}\end{figure}


\subsection{Attendance Request Status}
\label{\detokenize{manager/status-reporting:attendance-request-status}}
The attendance request status menu offers quick viewing of the past attendance requests made by their subordinates.

\begin{figure}[htbp]
\centering
\capstart

\noindent\sphinxincludegraphics[scale=0.5]{{manager-attendance-status}.PNG}
\caption{A attendance status report for managers.}\label{\detokenize{manager/status-reporting:id3}}\end{figure}

Clicking on the  \sphinxstyleemphasis{} icon opens the confirmation panel, with details about the request. You can also approve a yet unapproved request from this confirmation window.


\subsection{Leave Status}
\label{\detokenize{manager/status-reporting:leave-status}}
The leave request status menu offers quick viewing of the past leave requests made by their subordinates.

\begin{figure}[htbp]
\centering
\capstart

\noindent\sphinxincludegraphics[scale=0.5]{{manager-leave-status}.PNG}
\caption{A leave status report for managers.}\label{\detokenize{manager/status-reporting:id4}}\end{figure}

Clicking on the  \sphinxstyleemphasis{} icon opens the confirmation panel, with details about the request. You can also approve a yet unapproved request from this confirmation window.


\subsection{Loan Status}
\label{\detokenize{manager/status-reporting:loan-status}}
The loan request status menu offers quick viewing of the past loan requests made by their subordinates.

\begin{figure}[htbp]
\centering
\capstart

\noindent\sphinxincludegraphics[scale=0.5]{{manager-loan-status}.PNG}
\caption{A loan status report for managers.}\label{\detokenize{manager/status-reporting:id5}}\end{figure}

Clicking on the  \sphinxstyleemphasis{} icon opens the confirmation panel, with details about the request. You can also approve a yet unapproved request from this confirmation window.


\subsection{Travel Status}
\label{\detokenize{manager/status-reporting:travel-status}}
The travel request status menu offers quick viewing of the past travel requests made by their subordinates.

\begin{figure}[htbp]
\centering
\capstart

\noindent\sphinxincludegraphics[scale=0.5]{{manager-travel-status}.PNG}
\caption{A travel status report for managers.}\label{\detokenize{manager/status-reporting:id6}}\end{figure}

Clicking on the  \sphinxstyleemphasis{} icon opens the confirmation panel, with details about the request. You can also approve a yet unapproved request from this confirmation window.


\subsection{Day off Work Status}
\label{\detokenize{manager/status-reporting:day-off-work-status}}
The work on day off request status menu offers quick viewing of the past work on day off requests made by their subordinates.

\begin{figure}[htbp]
\centering
\capstart

\noindent\sphinxincludegraphics[scale=0.5]{{manager-work-dayoff-status}.PNG}
\caption{A work on day off status report for managers.}\label{\detokenize{manager/status-reporting:id7}}\end{figure}

Clicking on the  \sphinxstyleemphasis{} icon opens the confirmation panel, with details about the request. You can also approve a yet unapproved request from this confirmation window.


\subsection{Holiday Work Status}
\label{\detokenize{manager/status-reporting:holiday-work-status}}
Work on Holiday(WOH) request status menu offers quick viewing of the past WOH requests made by their subordinates.

\begin{figure}[htbp]
\centering
\capstart

\noindent\sphinxincludegraphics[scale=0.5]{{manager-woh-status}.PNG}
\caption{Work on holiday status report for managers.}\label{\detokenize{manager/status-reporting:id8}}\end{figure}

Clicking on the  \sphinxstyleemphasis{} icon opens the confirmation panel, with details about the request. You can also approve a yet unapproved request from this confirmation window.


\subsection{Training Status}
\label{\detokenize{manager/status-reporting:training-status}}
The training request status menu offers quick viewing of the past training requests made by their subordinates.

\begin{figure}[htbp]
\centering
\capstart

\noindent\sphinxincludegraphics[scale=0.5]{{manager-training-status}.PNG}
\caption{Training status report for managers.}\label{\detokenize{manager/status-reporting:id9}}\end{figure}

Clicking on the  \sphinxstyleemphasis{} icon opens the confirmation panel, with details about the request. You can also approve a yet unapproved request from this confirmation window.


\subsection{Overtime Status}
\label{\detokenize{manager/status-reporting:overtime-status}}
The overtime request status menu offers quick viewing of the past training requests made by their subordinates.

\begin{figure}[htbp]
\centering
\capstart

\noindent\sphinxincludegraphics[scale=0.5]{{manager-overtime-status}.PNG}
\caption{Overtime status report for managers.}\label{\detokenize{manager/status-reporting:id10}}\end{figure}

Clicking on the  \sphinxstyleemphasis{} icon opens the confirmation panel, with details about the request. You can also approve a yet unapproved request from this confirmation window.


\section{Appraisal Flow}
\label{\detokenize{manager/appraisal:appraisal-flow}}\label{\detokenize{manager/appraisal::doc}}
The appraisal portal for managers lets a manager quickly perform various tasks related to appraisal flow. It provides panels to give feedback, answer questionnaire as set in the appraisal evaluation, review and final review process.



\renewcommand{\indexname}{Index}
\printindex
\end{document}